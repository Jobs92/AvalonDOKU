%!TEX root = ../dokumentation.tex

\pagestyle{empty}

\renewcommand{\abstractname}{Abstract}
\begin{abstract}
Mitfahrgelegenheiten erleben momentan eine starke Nachfrage. Besonders kleine Unternehmen und Start-ups haben diesen Trend erkannt und entsprechende Cloud-Lösungen auf den Markt gebracht. Durch schlanke Strukturen und flache Hierarchien können diese sehr schnell neue Funktionen an den Kunden bringen. Große Unternehmen, wie beispielsweise SAP, können durch ihre Größe dieses Tempo nicht mitgehen und sind damit im Bereich der Cloud-Lösungen im Nachteil. \\

Im Rahmen dieser Projektarbeit werden Optimierungen durchgeführt, um diesen Nachteil auszugleichen. Dazu wird der Release-Prozess der SAP Mitfahrlösung TwoGo analysiert und den zentralen Merkmalen von \acl{CD} gegenübergestellt. Aus dieser Gegenüberstellung werden potentielle Optimierungsmöglichkeiten für den Release-Prozess von SAP TwoGo abgeleitet. Diese betreffen sowohl technische als auch organisatorische Aspekte.\\

Eine konkrete Optimierungsmöglichkeit ist der Deployment-Test. Dieser überprüft den Deployment-Vorgang und erkennt dabei Fehler, Warnungen und Erfolge. Zur Umsetzung dieses Deployment-Tests wird der Deployment-Vorgang analysiert, ein Entwurf erstellt und dieser implementiert und getestet.  
\end{abstract}
