%!TEX root = ../dokumentation.tex

\chapter{Planung und Vorgehen}

Bei der Planung und Entwicklung von Avalon soll Effektivität und Zielstrebigkeit eine wichtige Rolle spielen. Deshalb war es uns von Beginn an wichtig, genügend Zeit in die Planung zu investieren. Während der ersten Planungsphase haben wir entschieden, das Projektmanagement softwareseitig zu Unterstützen um die Organisation und die  Zusammenarbeit untereinander zu vereinfachen sowie den zeitlichen Aufwand für die Organisation zu minimieren.\\
 Außerdem haben wir uns entschieden die Entwicklung nach den Grundsätzen der agilen Softwareentwicklung durchzuführen, die wir in Semester 2 in einer Vorlesungen kennengelernt haben. Ein Grundbaustein der agilen Softwareentwicklung ist es zu Beginn zunächst ein Fachkonzept zu erstellen und im weiteren Verlauf der Entwicklung mehrere Iterationen durchzuführen. Mit diesen Iterationen sollen Fehler leichter aufgedeckt werden und Prioritäten für zentrale Aspekte von Avalon gesetzt werden können.

\section{Projektmanagement}

\subsection{Anforderungen}

Für die softwareseitige Unterstützung des Projektmanagements haben wir zunächst Anforderungen herausgearbeitet. Für uns ergaben sich drei zentrale Anforderungen an diese Software. \\
Die erste Anforderung ist es einen Gesamtüberblick über den aktuellen Stand des Projektes darzustellen. Dazu soll eine Aufgabenliste angezeigt werden, bei der sich Aufgaben auch an einzelne Personen zuordnen lassen. Diese Aufgabenliste soll einem Backlog entsprechen, wie man es aus dem Scrum-Prozess kennt.\\
Die zweite Anforderung ist eine zentrale Codeverwaltung. Damit soll erreicht werden das die Teammitglieder immer Zugriff auf den aktuellen Code haben und Erkennbar ist wer welche Veränderungen vorgenommen hat. Zudem soll die Codeverwaltung auch eine Versionsverwaltung bieten um gegen Datenverluste jeglicher Art vorzubeugen.\\
Die dritte Anforderung bezieht sich auf die Anforderungen 1 und 2. Da wir oft von unterschiedlichen Orten und Rechnern arbeiten, sollen die Daten überall und jederzeit verfügbar sein. Um diese Anforderung zu erfüllen lag es nahe, eine Lösung zu nutzen, bei der die Daten, online, in einer Cloud gespeichert werden.
Aufgrund von persönlichen Erfahrungen aus vergangenen Projekten und Internet Recherchen haben wir entschieden, dass wir zwei verschiedene Dienste einsetzten um diese Anforderungen abzudecken. Für den Aufgabenübersicht werden wir den Online-Dienst \textit{Trello} einsetzen und für die Codeverwaltung \textit{GitHub} benutzen. Beide Tools erfüllen die oben genannten Anforderungen vollständig.


\subsection{Einrichtung von Trello und GitHub}
 Die Einrichtung von Trello ist sehr einfach, da keine Software installiert werden muss und alles im Browser läuft. Zudem gibt es auch eine App für Smartphones, mit der man auch Unterwegs immer einen Überblick bekommen kann. Zur Einrichtung von Trello ist nur eine Registrierung auf der Webseite und die Erstellung eines gemeinsamen \textit{Boards} notwendig. Ein solches Board besteht aus beliebig vielen Spalten, denen die konkreten Aufgaben untergeordnet sind. Die Spalten dienen dazu das ganze Board zu gliedern. Am besten lässt sich dieses Prinzip anhand der folgenden Tabelle verstehen, welche den Aufbau unseres Boards beispielhaft darstellt.\\
\begin{center}
\begin{tabular}{|c|c|c|c|}\hline
  \textbf{ To-Do} & \textbf{Doing} & \textbf{Done} & \textbf{Idea's} \\ \hline
   Kaufalg. & Use-Cases & Marktanalyse & Historie \\ \hline
   ... & ... & ... & ...\\ \hline
 \end{tabular}
\end{center}
Auch die Einrichtung von GitHub ist recht einfach. Auch hier müssen sich die Mitglieder auf der GitHub Homepage registrieren. Danach erstellt ein Mitglied ein neues Repository und fügt die anderen Mitarbeiter als Collaborators hinzu. Zur Nutzung ist es noch notwendig \textit{Git} oder die \textit{GitHub-GUI} auf den einzelnen Rechnern zu installieren. Dieses Tool benötigt man um das aktuelle Repository herunterzuladen und Änderungen als Commit hochzuladen. Da in unserem Team nur sehr geringe Erfahrungen mit Git vorhanden waren, entschieden wir uns nur die GitHub-GUI zu nutzen.

\subsection{Praktische Erfahrungen mit Trello und GitHub}

Während Entwicklung von Avalon stellte sich besonders Trello als sehr hilfreiches, einfaches und fehlerfreies Tool dar. Die Funktionsweise wurde jedem sofort verständlich und war selbst erklärend. Mit diesen Eigenschaften konnte Trello immer einen guten Überblick über den aktuellen Stand liefern und erfüllte die von uns gestellten Anforderungen sehr gut.\\
GitHub dagegen lief nicht ganz so unproblematisch. Die Funktionsweise über die GUI war zwar sehr logisch und einleuchtend, allerdings kam es regelmäßig zu teils merkwürdigen Fehlern. Das größte Problem entstand, wenn zwei Mitglieder die gleiche Datei bearbeiteten. Das Zusammenführen der beiden Änderungen (engl. mergen) über die GUI war nie möglich und manuelles Mergen über die Konsole funktionierte nur selten. Um dieses Problem im praktischen Einsatz schnell und einfach zu Lösen, hat einer der beiden Betreffenden seine Änderungen separat gespeichert und das Repositroy neu heruntergeladen. Diese Fehler führte oft zu Frustrationen innerhalb des Teams. Der Grund für diese Fehler lag sowohl an unsere Unerfahrenheit mit Codeverwaltung und an der GitHub GUI, da diese anscheinend bestimmte Befehle nicht korrekt ausführt.\\
Ein weiterer frustrierender Fehler war das Überschreiben von einigen Dateien mit älterem Inhalt. Der Grund hierfür lag an der falschen Konfiguration unseres Repositroys. Allerdings konnten wir aus Zeitgründen diesen Fehler nicht beheben.\\
All diese Probleme führten immer wieder zu Überlegungen GitHub durch ein anderes Tool, wie Dropbox, zu ersetzen. Diese Überlegungen wurden jedoch wieder verworfen, da damit vermutlich noch weitere Probleme entstanden wären. Im Nachhinein wäre es besser gewesen die Git-Bash zu verwenden und vor Projektbeginn eine ausführliches Tutorial zu Git zu machen.

\section{Fachkonzept und Iterationen}

Aufgrund der Zeitspanne von drei Monaten über welche dieses Projekt verlief, haben wir uns entschieden zunächst ein Fachkonzept zu erstellen und darauf aufbauend drei Iterationen durchzuführen. Nach den Prinzipien der agilen Softwaremethodik sind die Übergänge zwischen diesen Iterationen in beide Richtungen offen. Dies war für uns sehr essentiell, um Ergänzungen vorzunehmen  und um bestimmte Aspekte im Nachhinein noch konkretisieren zu können.

\subsection{Erstellung des ersten Fachkonzepts}
Bevor wir mit den Iterationen beginnen konnten, war es zunächst notwendig das Fachkonzept zu planen und zu erstellen. Als Grundlage für diese wurde zunächst eine Marktanalyse durchgeführt, um die zentralen Merkmale des Smartphone-Marktes herauszuarbeiten. Hierbei spielten auch unsere eigene Erfahrungen mit Smartphones im Unternehmen und im privaten Umfeld eine entscheidende Rolle. Aus diesen Erkenntnissen konnte ein erstes Modell unseres Spiel grob skizziert werden und die grundlegende Spielidee festgemacht werden. Diese Erkenntnisse wurde in Form eines Pflichtenheft festgehalten und führten zur Erstellung der ersten Use-Cases. \\
Aus dem Pflichtenheft und den Use-Cases ließen sich die ersten Klassen identifizieren. Dabei wurden nach den Regeln der Objekt Orientieren Analyse (kurz: OOA) nur Klassen identifiziert, die für die Logik des späteren Spiels notwendig sind. Während dieser Phase ließen sich auch schon einige Assoziationen und Vererbungsstrukturen, wie zum Beispiel bei den Department Klassen, erkennen. Mit diesem grundlegenden Fachkonzept war es möglich den ersten Iterationsschritt einzuleiten. Während den Iterationen wurde das Fachkonzept an einigen Stellen noch um einige Klassen, Assoziationen und Vererbungen ergänzt und präzisiert.\\
Zeitlich wurde das erste Fachkonzept innerhalb der ersten zwei Wochen des Projektes erstellt.


\subsection{1. Iteration}
Die erste Iteration hatte das Ziel die zentralen Merkmale des Fachkonzept zu implementieren und somit ein Grundgerüst für die weitere Entwicklung aufbauen. Dazu wurden die wichtigsten Klassen aus dem Fachkonzept, wie zum Beispiel die Departments in Java programmiert. Komplexere Klassen, wie der Markt, wurden dabei bewusst ohne Inhalt implementiert. Während der Durchführung der ersten Iteration sind uns immer wieder einige Lücken und Fehler im Fachkonzept aufgefallen, die ein funktionierendes Spiel verhindert haben. Um die Vollständigkeit und die Richtigkeit zu gewährleisten haben wir in diesen Fällen das Fachkonzept um die fehlenden Merkmale ergänzt beziehungsweise die Fehler behoben. \\
Um nicht den Überblick zu verlieren wurde Parallel zur Implementierung der Klassen des Fachkonzepts bereits, soweit wie möglich, die ersten Unit-Tests geschrieben. Dieser Schritte hatte die Ambition, zu gewährleisten, dass der Quellcode fehlerfrei ist. Nachdem die wichtigsten Klassen implementiert waren, wurde der zweite Iterationsschritt eingeleitet.

\subsection{2. Iteration}
Im zweiten Iterationsschritt wurden weitere Klassen aus dem Fachkonzept implementiert. Besonders wurde die Implementierung des Market und der Consumer Groups priorisiert, da diese ein sehr wichtiges Element von Avalon sind. Während dieser Phase wurde auch eine Excel-Tabelle erstellt, welche den Kaufalgorithmus in seinem vollen Umfang simuliert. Damit konnte die Funktionsweise des Algorithmus bereits vor der Implementierung sichergestellt und mit unterschiedlichen Parametern getestet werden. \\
Ein zweiter wichtiger Punkt in dieser Iteration war die Implementierung des Client/Server Modells und der Config-Datei. Somit ging diese Iteration über das Fachkonzept hinaus und fügte weitere Mechanismen hinzu, die für das fertige Spiel zwingend notwendig sind. Auch während dieser Iteration wurden, soweit wie möglich, Unit-Tests geschrieben um mögliche Fehler zu entdecken und zu beheben. 

\subsection{3. Iteration}
Die dritte und letzten Iteration hatte das Ziel die noch fehlenden Klassen des Fachkonzepts und die Benutzeroberfläche zu implementiert, sowie Fehler zu beheben. Zu den fehlenden Klassen des Fachkonzepts handelte es sich konkret, um die Zufallsereignisse für Einzelunternehmen und alle Unternehmen. Die Benutzeroberfläche wurde in kleinen \glqq Unter- Iterationen\grqq  von einem ersten Prototyp bis hin zur vollständigen Oberfläche umgesetzt. Dabei war immer das Design der Mockup's das Ziel. Damit die Benutzeroberfläche funktionieren konnte, muss auch noch eine Schnittstelle implementiert werden, die den Datentransfer zwischen Fachkonzept, Kommunikation und Benutzeroberfläche sicherstellt.\\
Schließlich stand in der letzten Iteration noch die Fehlerbehebung an. Dazu wurden die noch fehlenden Unit-Tests implementiert und ausgeführt. Zudem wurde das Spiel unter realistischen Umständen gespielt. Durch diesen praktischen Test konnte größtenteils Fehler gefunden werden, die außerhalb des Fachkonzepts waren.\\
Durch diese zwei unabhängigen Testmethoden konnte gewährleistet werden, dass das Spiel den Qualitätskriterien entspricht und damit verbunden auch spielbar ist.

\subsection{Reflexion der Iterationsschritte}
Die Einführung der Iterationsschritte war sehr Sinnvoll, da das Projekt so in kleine Teiletappen eingeteilt werden konnte. Die Übergänge in den Iterationsschritten waren bei uns sehr fließend. So kam es vor, dass ein Teammitglied bereits den nächsten Iterationsschritt bearbeitete. Mit diesen flexiblen Übergängen und den Iterationen konnten einzelne Aufgaben auch gut an die einzelnen Teammitglieder und deren Fähigkeiten zugeordnet werden. Zudem konnte durch die genaue definierten Ziele der Iterationsschritte immer ein Gesamtstatus des Projektes gewährleistet werden. Dies war eine sehr wichtige Entscheidungsbasis für langfristige Entscheidungen.\\
Zusammenfassend lässt sich sagen, dass die Effektivität der Entwicklung durch die Iterationsschritte gesteigert wurde und ohne diese Schritte wohl kein zufriedenstellendes Ergebnis entstanden wäre.

