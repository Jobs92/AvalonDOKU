%!TEX root = ../dokumentation.tex

\chapter{Anforderungsspezifikation}

\section{Zielbestimmung}
Der Spieler soll ein Unternehmen steuern, dass Smartphones produziert und an Endkunden verkauft.
\subsection{Muss-Kriterien}
\begin{itemize}
\item Steuerung der Abteilungen: Einkauf, Produktion, Verkauf, Forschung, Marketing und Rechtsabteilung

\item Spieler spielen gegen andere menschliche Spieler.

\item Die Nachfrage nach Smartphones soll nach realistischen Vorgaben simuliert werden.

\item Das Spiel läuft rundenbasiert ab. Eine Runde entspricht einer Periode.

\item Spieler können über ein Netzwerk spielen, kein Hot-Seat!

\item Abteilungen und Produkte können durch ein "Level up" verbessert werden.
\item Zufallsereignisse sollen den Spielverlauf beeinflussen.
\end{itemize}

\subsection{Kann-Kriterien}

\begin{itemize}
\item  Spiele Lobby mit Chat
\item Downgrade von Abteilungen um Kosten einzusparen.
\end{itemize}

\subsection{Abgrenzungskriterien}
Aufgabe des Spieles ist es nicht sämtliche Aspekte eines Unternehmens zu simulieren. Es genügt wenn wesentliche Merkmale eines Smartphones-Herstellers abgebildet werden.

\section{Einsatz}
Die Zielgruppe dieses Unternehmensplanspiel sind Studenten und anderen Personen, die Interesse haben ein Unternehmen zu steuern, das Smartphones herstellt. Der Anwendungsbeireich spielt keine Rolle. Da die Software regelmäßig Eingaben benötigt, ist ein unbeaufsichtigter Betrieb nicht vorgesehen.

\section{Umgebung}
Die Software soll auf Windows-Rechnern laufen, auf denen einen aktuelle Java Version installiert ist. Die Hardware muss den Java Mindestanforderungen entsprechen und es muss eine Netzwerkkarte vorhanden sein.

\section{Funktionalität}
Typische Abläufe im Planspiel sind:
\begin{itemize}
\item Das Einkaufen von Rohstoffen.
\item Das Produzieren von Smartphones.
\item Das Verkaufen von Smartphones.
\item Das Upgraden von Abteilungen.
\item Das Starten von Forschungen und Spionagen.
\item Das Starten von Marketingkampagnen.
\item Das Überprüfen und Verklagen von anderen Spieler. 
\item Das Anfechten eines Rechtsstreites.
\end{itemize}

\section{Daten}
Benutzereingaben müssen nicht dauerhaft gespeichert werden. Die Startparameter für das Spiel sollen in einer Datei gespeichert werden.

\section{Leistungen}
Das Spiel soll zügig auf Eingaben reagieren. Wartezeiten sollen unter 10 Sekunden liegen.

\section{Benutzeroberfläche}
Die Benutzeroberfläche soll alle veränderbaren und nicht veränderbaren Parameter des Unternehmens darstellen. Zudem soll die Oberfläche minimalistisch und übersichtlich sein. 

\section{Qualitätsziele}
Fehlerfreies Spielen des Spiels und gute Benutzbarkeit.

\section{Ergänzungen}

