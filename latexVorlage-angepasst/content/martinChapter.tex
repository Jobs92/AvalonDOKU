%!TEX root = ../dokumentation.tex

\chapter{Planung und Vorgehen}

Bei der Planung und Entwicklung von Avalon soll Effektivität und Zielstrebigkeit eine wichtige Rolle spielen. Deshalb war es uns von Beginn an wichtig, genügend Zeit in die Planung zu investieren. Während der ersten Planungsphase haben wir entschieden, das Projektmanagement softwareseitig zu Unterstützen um die Organisation und die  Zusammenarbeit untereinander zu vereinfachen sowie den zeitlichen Aufwand für die Organisation zu minimieren. Außerdem haben wir entschieden Iterationen bei der Entwicklung durchzuführen, da damit Fehler leichter aufgedeckt werden und Prioritäten  für wichtige Aspekte gesetzt werden können.

\section{Projektmanagement}

\subsection{Anforderungen}

Da es sich bei Avalon um ein sehr komplexes Projekt handelt, soll  Zur Unterstützung des Entwicklungsprozesses haben wir uns entschieden Programme einzusetzen, die Aufgaben des Projektmanagements abnehmen bzw. erleichtern.
Für uns ergaben sich zwei zentrale Anforderungen an diese Programme. Die erste Anforderung ist es einen Gesamtüberblick über den aktuellen Stand des Projektes darzustellen. Dazu gehört, was  noch erledigt werden muss, was  gerade gemacht wird und was  bereits erledigt wurde.
Die zweite Anforderung ist eine zentrale Codeverwaltung mit Versionsverwaltung, um das gemeinsame und zeitgleiche Programmierung zu ermöglichen. Außerdem soll durch Versionsverwaltung gegen Datenverluste vorgebeugt werden.\\
Da wir oft von unterschiedlichen Orten und Rechnern arbeiten, müssen die genannten Anforderungen immer und überall erfüllt werden. Um diese, weitere Anforderung zu erfüllen lag es nahe, die Daten in der Cloud zu speichern.\\
Aufgrund von persönlichen Erfahrungen und Internet Recherchen haben wir entschieden, dass wir für den Gesamtüberblick \textit{Trello} benutzen und für die Codeverwaltung \textit{GitHub} benutzen. Beide Tools erfüllen die oben genannten Anforderungen.


\subsection{Einrichtung von Trello und GitHub}
 Die Einrichtung von Trello ist sehr einfach, da keine Software installiert werden muss und alles im Browser läuft. Zur Einrichtung ist nur eine Registrierung und die Erstellung eines gemeinsamen \textit{Boards} notwendig. Ein solches Board besteht aus beliebig vielen Spalten, in denen die konkreten Aufgaben stehen. Unser Avalon Board hat folgenden Aufbau:\\
\begin{center}
\begin{tabular}{|c|c|c|}\hline
  \textbf{ To-Do} & \textbf{Doing} & \textbf{Done} \\ \hline
   Kaufalg. & Use-Cases & Marktanalyse \\ \hline
   ... & ... & ... \\ \hline
 \end{tabular}
\end{center}
Auch die Einrichtung von GitHub ist recht einfach. Auch hier müssen sich die Mitglieder auf der GitHub Homepage registrieren. Danach erstellt ein Mitglied ein neues Repository und fügt die anderen Mitarbeiter als Collaborators hinzu. Zur Nutzung ist es noch notwendig \textit{Git} oder die \textit{GitHub-GUI} auf den einzelnen Rechnern zu installieren. Dieses Tool benötigt man um das Repository herunterzuladen und Änderungen hochzuladen. Da in unserem Team nur geringe Erfahrungen mit Git vorhanden waren, entschieden wir uns die GitHub-GUI zu nutzen.
\subsection{Praktische Erfahrungen mit Trello und GitHub}

Während Entwicklung von Avalon stellte sich besonders Trello als sehr hilfreiches, einfaches und fehlerfreies Tool dar. Die Funktionsweise wurde jedem sofort verständlich und war selbst erklärend. Mit diesen Eigenschaften konnte Trello immer einen guten Überblick über den aktuellen Stand liefern und erfüllte die von uns gestellten Anforderungen sehr gut.\\
GitHub dagegen lief nicht ganz so unproblematisch. Die Funktionsweise über die GUI war zwar sehr logisch und einleuchtend, allerdings kam es regelmäßig zu teils merkwürdigen Fehlern. Das größte Problem entstand, wenn zwei Mitglieder die gleiche Datei bearbeiteten. Das Mergen der Datei über die GUI war nie möglich und manuelles Mergen über die Konsole funktionierte nur selten. Um dieses Problem schnell und einfach zu Lösen, hat einer der beiden seine Änderungen separat gespeichert und das Repositroy neu heruntergeladen. Diese Fehler führte oft zu Frustrationen. Der Grund für diese Fehler lag sowohl an unsere Unerfahrenheit mit Codeverwaltung und an der GitHub GUI. Ein weiterer frustrierender Fehler war das Überschreiben von einigen Dateien mit älterem Inhalt. All diese Probleme führten zu Überlegungen GitHub durch ein anderes Tool, wie Dropbox, zu ersetzen. Diese Überlegungen wurden jedoch wieder verworfen, da damit vermutlich noch weitere Probleme entstanden wären. Im Nachhinein wäre es besser gewesen die Git-Bash zu verwenden und eine ausführliches Tutorial zu Git zu machen.




