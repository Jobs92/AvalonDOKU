%!TEX root = ../dokumentation.tex

\chapter{Zusammenfassung und Ausblick}
Ziel dieser Projektarbeit war es, den Release-Prozess von SAP TwoGo zu optimieren. Primär sollte dabei der Faktor Zeit durch das Aufzeigen von Optimierungsmöglichkeiten und die Entwicklung eines Deployment-Tests optimiert werden. \\
Um dieses Ziel zu erreichen, wurde zunächst die Infrastruktur und der aktuelle Deployment Prozess von SAP TwoGo betrachtet. Dabei wurden sowohl die technischen als auch die organisatorischen Aspekte erörtert. Um die Entwicklung des Deployment-Tests zu ermöglichen, wurde Basiswissen über Unix-Systeme und Shell-Programmierung dargestellt. Zudem wurden die zentralen Merkmale von \acs{CD} in Verbindung mit \acs{CI} erläutert. Mit diesem Hintergrund konnten die Vor- und Nachteile von \acs{CD} herausgearbeitet werden.\\
Aus diesen Themen konnte danach ein Vergleich zwischen \acs{CD} und dem aktuellen Deploymentprozess durchgeführt werden. Das Resultat dieses Vergleiches war, dass bereits viele Aspekte von \acs{CD} umgesetzt wurden, wie zum Beispiel die Verwendung eines \acs{CI} Systems. Es haben sich jedoch auch noch fehlende Aspekte gezeigt, wie beispielsweise der Deployment-Test. Dessen Entwicklung wurde ausführlich in Kapitel 4 beschrieben.\\
Zusammenfassend lässt sich  sagen, dass viele Optimierungsmöglichkeiten gefunden wurden und eine dieser Möglichkeiten konkret umgesetzt wurde. Damit bleiben jedoch noch viele weitere Optimierungsmöglichkeiten offen, wie beispielsweise dem einmaligen Kompilieren der Quelldateien. Daraus ergibt sich, dass momentan ein schnelles Deployment nach dem Prinzip von \acs{CD} nicht möglich ist. Die Gründe hierfür sind primär organisatorischer Herkunft, da die internen Prozesse und Strukturen innerhalb der SAP ein schnelleres Deployment verhindern.\\
Für die Zukunft und die damit verbundene vollständige Umsetzung von \acs{CD} ist es notwendig, Veränderungen an diesen Prozessen und Strukturen vorzunehmen. Zudem müssen auch noch einige technische Veränderungen durchgeführt werden, wie dem einmaligen Bauen von Binärdateien.\\
In der Summe sind bereits viele Teile des \acs{CD} Puzzles am richtigen Platz, fehlende Teile konnten identifiziert werden und eines dieser fehlenden Teile wurde mit dem Deployment-Test eingefügt.
